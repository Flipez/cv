\documentclass[11pt,a4paper,nolmodern]{moderncv}

\title{Site Reliability Engineer}

\usepackage{flipez}

\newcommand{\car}{Führerschein Klasse B}
\newcommand{\birthday}{05.06.1994}
%% start of file `template.tex'.
%% Copyright 2006-2010 Xavier Danaux (xdanaux@gmail.com).
%
% This work may be distributed and/or modified under the
% conditions of the LaTeX Project Public License version 1.3c,
% available at http://www.latex-project.org/lppl/.

% Version: 20110122-4


\usepackage[onehalfspacing]{setspace}
\usepackage{fontspec}
\usepackage[english]{babel}
\usepackage{genealogytree}
\linespread{1.18}
% for some reason, lines take up a lot of space in itemize in English...
\newenvironment{tightitemize}
   {\begin{itemize}
   \setlength{\parskip}{0pt}}
   {\end{itemize}}


% personal data
\extrainfo{%
\faXing~\httplink{xing.com/profile/Robert\_Mueller349}\\%
\faGithub~\httplink{github.com/flipez}\\%
\faCar~\car\\%
\gtrsymBorn~\birthday}

\myquote{It's not about the stack. It's about the people}{}

%\nopagenumbers{}                             % uncomment to suppress automatic page numbering for CVs longer than one page
\begin{document}
\setmainfont{TeX Gyre Pagella}
%\setsansfont{Myriad Pro}

\hyphenpenalty=10000
%\maketitle



\maketitle

\section{Fähigkeiten}

\subsection{Kernfähigkeiten}
\cvcomputer{Entwicklung}{Ruby, Ruby on Rails}
           {Operativer Bereich}{Ubuntu, GitLab (CI), Docker, Kubernetes}
\vskip 0.1in

\subsection{Entwicklung}
\cvcomputer{Sprachen}{Ruby, JavaScript, Crystal, Shell/Bash}
           {Frameworks}{Rails, RSpec, Rubocop, Capybara, Selenium, React, Sinatra}
\vskip 0.1in
\cvcomputer{Datenbanken}{Redis, PostgreSQL}
           {Methoden}{Objektorientierte Programmierung, MVC}
\vskip 0.1in
\cvcomputer{Source Management}{Git, Flux}
           {Werkzeuge}{GitLab, GitHub, GitLab CI, Ansible}
\vskip 0.1in

\subsection{System- und Infrastruktur-Administration}
\cvcomputer{Web}{Nginx, HAProxy, BGP}
           {Monitoring}{Sentry, Zabbix, Munin, Prometheus, Grafana}
\vskip 0.1in
\cvcomputer{Backup}{Duply, Borg}
           {\kern-1em Virtualisierung}{Docker}
\vskip 0.1in
\cvcomputer{Installation / Deployment}{Ansible, Cloudinit, systemd}
           {Infrastruktur Software}{Docker Registry, Minio}
\vskip 0.1in
\cvcomputer{\kern-3ex Betriebssysteme}{Debian, Ubuntu, Alpine}
           {Datenbanken}{Redis, PostgreSQL}
\vskip 0.1in

\section{Fremdsprachen}

\cvlanguage{Englisch}{Fließend, Tägliche Nutzung}{}

\newpage

\section{Erfahrung}

\tlcventry{2019}{0}{Site Reliability Engineer}{\href{https://www.hetzner-cloud.de/}{Hetzner Cloud GmbH}}{Unterföhring}{}%
  {
\begin{itemize}
  \item Entwicklung und Design neuer Features der Hetzner Cloud
  \item Konzeption und Umsetzung des Loadbalancer Produktes
  \begin{itemize}
    \item Planung der technischen Umsetzung im Operativen Bereich
    \item Konzeption der Architektur des Soft- und Hardwarestacks
    \item Betreuung und Weiterentwicklung der Infrastruktur
  \end{itemize}
  \item Betreuung mehrerer Kubernetes Cluster für internen Gebrauch
  \item Betrieb der firmenweiten GitLab Instanz; Product Owner
  \item Weiterentwicklung des Tools für internes Vorschlagswesen; Product Owner
  \item Organisation interner Entwicklertreffen zum Wissensaustausch und der Kommunikationsverbesserung
  \item Unterstützung bei Planung und Organisation teaminterner Aufgaben und Projekte
\end{itemize}}
\tlcventry{2016}{2019}{Sofware Engineer, Systems Engineer}{\href{https://www.hetzner.com/}{Hetzner Online GmbH}}{Falkenstein}{}%
  {
\begin{itemize}
  \item Software Engineer:
    \begin{itemize}
      \item Eigenständige Entwicklung und Betrieb interner Web-Applikationen für die Administration
            interner Dienste
      \item Konzeption und Entwicklung eines ISMS nach ISO 27001
        \begin{itemize}
          \item Entwicklung einer Web-Applikation zur Erfassung, Strukturierung und Auswertung der
                konzernweiten Assets
          \item Ausarbeitung, Umsetzung und Schulung von Richtlinien für Enwicklungsabläufe
        \end{itemize}
      \item Planung und Entwicklung einer Applikation für internes Projektmanagement und
            Vorschlagswesen.
      \item Konzept und Umsetzung von Verbesserungen interner Entwicklungs- und Deploymentprozesse
            sowie Vereinheitlichung der Standards
        \begin{itemize}
          \item Zielsetzung der Art des Deployments (Containerisiert)
          \item Evaluierung einheitlicher Codestyle Guidelines und Testrichtlinien
        \end{itemize}
    \end{itemize}
  \item Systems Engineer:
    \begin{itemize}
      \item Enwicklung und Betrieb der neuen Cloud-Platform mit verteiltem Netzwerkstorage auf Basis
            von Ceph und Virtualisierung mit Qemu
      \item Einfühung von Git als Firmenstandard zur Versionskontrolle; interne Schulungen und
            Beratung bei der Migration von SVN zu Git
      \item Evalution und Betrieb der firmenweiten GitLab Instanz; außerdem Product Owner
      \item Integration von CI/CD Methodiken in den Arbeitsablauf der Entwicklungsabteilungen
      \item Entwicklung und Betrieb einer Monitorlösung für Duply mit einer zentralen JSON-API
      \item Ausarbeitung, Entwicklung und Betrieb einer Static Site Deployment Lösung mit
            Integration in die GitLab CI Umgebung für automatisches Deployment
      \item Implementierung und Betrieb einer Firmenweiten Docker Registry mit Integration an das
            zentrale GitLab für CI und CD Prozesse.
      \item Konzeption und Betrieb einer Firmenweiten Storage Platform mit S3 Schnittstelle auf
            Basis von Minio für Integration für GitLab Runner, GitLab CI und interner Systeme.
      \item Evaluierung einer internen Containerplatform auf Basis von Openshift/Kubernetes
      \item Konzeption und Automatisierung interner Infrastruktur mit Ansible und Terraform
    \end{itemize}
\end{itemize}}

\newpage

\section{Projekte}

\tlcventry{2019}{0}{Entwickler und Maintainer}{\href{https://github.com/voxpupuli/vox-pupuli-tasks}{Vox Pupuli Tasks}}{}{}%
{
\begin{itemize}
  \item "Vox Pupuli Tasks - The Webapp for community management"
    \begin{itemize}
      \item Konzeption und Koordination
      \item Entwicklung und Design neuer Features
    \end{itemize}
\end{itemize}}

\section{Ausbildung}

\tlcventry{2013}{2016}{Fachinformatiker Anwendungsentwicklung}{\href{https://hetzner.com/}{Hetzner Online GmbH}}{Falkenstein}{}{
\begin{tightitemize}
  \item Ausbildung zum Fachinformatiker Systemintegration
  \item Wechsel zum Fachinformatiker Anwendungsentwicklung im dritten Jahr
  \item Rotation durch die Abteilungen Rechenzentrums-Verkabelung, Hardwarebau, Server-wartung, Serversupport, Rechenzentrums-Softwareentwicklung, Produktentwicklung und Technologieforschung
  \item Verwaltung, Entwicklung und Optimierung der vorhandenen Virtualisierungsplatform mit über 50.000 virtuellen Instanzen
  \item Planung und Evaluierung einer neuen Cloud-Platform mit verteiltem und lokalem Storage
\end{tightitemize}}

\tlcventry{2010}{2013}{Industriekaufmann}{\href{https://best-gmbh.net/}{best gmbh marketing services}}{Glaubitz}{}{
\begin{tightitemize}
  \item Erstellung von Angeboten, Verträgen und Rechnungen
  \item Erfahrungen im persönlichen Kundenkontakt
  \item Koordination Abteilungsübergreifender Kommunikation und Logistik
\end{tightitemize}}

\end{document}
